\documentclass{article}

%\usepackage{amsmath}
%\usepackage{algorithmic}
%\usepackage{graphicx}
%\usepackage{xspace}

\begin{document}
\title{Hello LATEX World}

\author{Jucheol Moon}

\maketitle

\begin{abstract}
This document is a model and instructions for LATEX 'article' class.
\end{abstract}

\section{Introduction}
Welcome to the LATEX world.

\section{Ease of Use}

\subsection{Maintaining the Integrity of the Specifications}
The `article' class is used to format your paper and style the text. All margins, column widths, line spaces, and text fonts are prescribed.

\section{Styling Guide}

\subsection{Abbreviations and Acronyms}
Define abbreviations and acronyms the first time they are used in the text, 
even after they have been defined in the abstract.

\subsection{Equations}
\begin{equation}
1+1=2
\end{equation}

Taylor series in a text would be $1+1=2$.

\subsection{Lists}
Bullet style list.


Number style list.


\subsection{Figures and Tables}
\paragraph{Positioning Figures and Tables} Figure captions should be below the figures; table heads should appear above the tables. Insert figures and tables after they are cited in the text. Use the abbreviation .

\begin{table}
\begin{center}
%\begin{tabular}
%
%\end{tabular}
\end{center}
\end{table}

\begin{figure}

\end{figure}

\subsection{Algorithms}
%\begin{algorithmic}
% 
%\end{algorithmic}

\subsection{Source codes}
\begin{verbatim}

\end{verbatim}

\subsection{References}


\begin{thebibliography}{00}
\bibitem{Eason} G. Eason, B. Noble, and I. N. Sneddon, ``On certain integrals of Lipschitz-Hankel type involving products of Bessel functions,'' Phil. Trans. Roy. Soc. London, vol. A247, pp. 529--551, April 1955.
\bibitem{Maxwell} J. Clerk Maxwell, A Treatise on Electricity and Magnetism, 3rd ed., vol. 2. Oxford: Clarendon, 1892, pp.68--73.
\bibitem{Jacobs} I. S. Jacobs and C. P. Bean, ``Fine particles, thin films and exchange anisotropy,'' in Magnetism, vol. III, G. T. Rado and H. Suhl, Eds. New York: Academic, 1963, pp. 271--350
\end{thebibliography}

\end{document}
